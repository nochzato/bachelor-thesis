% Титульный лист
\linespread{1.1}

\begin{center}
    {\bfseries
        НАЦІОНАЛЬНИЙ ТЕХНІЧНИЙ УНІВЕРСИТЕТ УКРАЇНИ \par
        <<КИЇВСЬКИЙ ПОЛІТЕХНІЧНИЙ ІНСТИТУТ \par
        імені Ігоря СІКОРСЬКОГО>>\par
        Навчально-науковий фізико-технічний інститут\par
        Кафедра математичного моделювання та аналізу даних}
\end{center}
\par

\linespread{1.1}
Рівень вищої освіти --- перший (бакалаврський)

Спеціальність --- 113~Прикладна математика,

ОПП <<Математичні методи моделювання, розпізнавання образів та комп’ютерного зору>>

\vspace{10mm}
\begin{tabularx}{\textwidth}{XX}
     & ЗАТВЕРДЖУЮ                                                      \\[06pt]
     & В.о. завідувача кафедри                                         \\[06pt]
     & \rule{2.5cm}{0.25pt} Ганна ЯЙЛИМОВА                             \\[06pt]
     & <<\rule{0.5cm}{0.25pt}>> \rule{2.5cm}{0.25pt} \YearOfDefence~р.
\end{tabularx}

\vspace{5mm}
\begin{center}
    {\bfseries ЗАВДАННЯ \par}
    {\bfseries на дипломну роботу \par}
\end{center}

%%%%%====================================
% !!! Не чіпайте наступні три команди!
%%%%%====================================
\frenchspacing
\doublespacing          % інтервал "1,5" між рядками, тепер навічно
\setfontsize{14}

Студент: \reportAuthor \par

1. Тема роботи: <<\emph{\reportTitle}>>,
науковий керівник дисертації: \supervisorRegalia ~\supervisorFio, \par
затверджені наказом по університету \No 2251-С від <<31>> травня \YearOfDefence~р.

2. Термін подання студентом роботи: <<7>> червня \YearOfDefence~р.

% Коли будете заповнювати пункти 3-10, приберіть команду \emph --- вона тільки для виділення моїх коментарів
3. Об'єкт дослідження: процес виявлення кратерів за супутниковими знімками.

4. Предмет дослідження: методи виявлення кратерів, сформованих внаслідок вибухів
бомб чи інших вибухових пристроїв, з використанням сучасних технологій обробки зображень,
супутникового зондування та геоінформаційних систем.

5. Перелік завдань: провести огляд опублікованих джерел за тематикою дослідження,
формалізувати задачу сегментації кратерів на супутникових знімках
та адаптувати до неї відомі методи розв'язання задачі сегментації,
підготувати датасет для навчання моделей машинного навчання,
провести порівняльний аналіз методів розв'язання задачі сегментації
кратерів на супутникових знімках, ґрунтуючись на створеному датасеті.

6. Орієнтовний перелік графічного (ілюстративного) матеріалу: Презентація доповіді

7. Орієнтовний перелік публікацій: виступ на XXII Всеукраїнській
науково-практичній конференції студентів, аспірантів
та молодих вчених.

8. Дата видачі завдання: 30 вересня \YearOfBeginning~р.

% Якщо перша частина завдання вилізе за сторінку - приберіть команду \newpage
% Календарний план є продовженням завдання, а не окремою частиною

\begin{center}
    Календарний план
\end{center}

\renewcommand{\arraystretch}{1.5}
\begin{table}[h!]
    \setfontsize{14pt}
    \centering
    \begin{tabularx}{\textwidth}{|>{\centering\arraybackslash\setlength\hsize{0.25\hsize}}X|>{\setlength\hsize{2\hsize}}X|>{\centering\arraybackslash\setlength\hsize{1\hsize}}X|>{\centering\arraybackslash\setlength\hsize{0.75\hsize}}X|}
        \hline \No\par з/п                                          & Назва етапів виконання магістерської дисертації & Термін виконання & Примітка \\
        \hline
        % номер етапу
        1                                                           &
        % назва етапу
        Узгодження теми роботи із науковим керівником               &
        % термін виконання
        01-15 вересня \YearOfBeginning~р.                           &
        % примітка - зазвичай "Виконано"
        Виконано                                                                                                                                    \\
        %%% -- початок інтервалу для копіювання
        \hline
        % номер етапу
        2                                                           &
        % назва етапу
        Огляд опублікованих джерел за тематикою дослідження         &
        % термін виконання
        Вересень-листопад \YearOfBeginning~р.                       &
        % примітка - зазвичай "Виконано"
        Виконано                                                                                                                                    \\
        %%% -- кінець інтервалу для копіювання
        % не прибирайте амперсанди та \\ наприкінці рядків!
        % скопійовані інтервали вставляти перед фінальною \hline та заповнювати відповідно
        % ось так:
        %%% -- початок інтервалу для копіювання
        \hline
        % номер етапу
        3                                                           &
        % назва етапу
        Огляд моделей машинного навчання та функцій втрат           &
        % термін виконання
        Листопад \YearOfBeginning~р.-січень \YearOfDefence~р.       &
        % примітка - зазвичай "Виконано"
        Виконано                                                                                                                                    \\
        %%% -- кінець інтервалу для копіювання
        \hline
        % номер етапу
        4                                                           &
        % назва етапу
        Імплементація та тренування моделей машинного навчання      &
        % термін виконання
        Січень-квітень \YearOfDefence~р.                            &
        % примітка - зазвичай "Виконано"
        Виконано                                                                                                                                    \\
        \hline
        % номер етапу
        5                                                           &
        % назва етапу
        Проведення порівняльного аналізу моделей машинного навчання &
        % термін виконання
        Квітень-травень \YearOfDefence~р.                           &
        % примітка - зазвичай "Виконано"
        Виконано                                                                                                                                    \\
        \hline
        % номер етапу
        6                                                           &
        % назва етапу
        Оформлення дипломної роботи                                 &
        % термін виконання
        Травень-14 червня \YearOfDefence~р.                         &
        % примітка - зазвичай "Виконано"
        Виконано                                                                                                                                    \\
        \hline %фінальна hline
    \end{tabularx}
\end{table}

\renewcommand{\arraystretch}{1}
\begin{tabularx}{\textwidth}{>{\setlength\hsize{1.2\hsize}}X >{\setlength\hsize{0.5\hsize}}X >{\setlength\hsize{1.3\hsize}}X}
    Студент  & \rule{2.5cm}{0.25pt} & \reportAuthorShort  \\[06pt]
    Керівник & \rule{2.5cm}{0.25pt} & \supervisorFioShort \\
\end{tabularx}

\newpage
