%!TEX root = ../thesis.tex
% створюємо Висновки до всієї роботи
Отже, у даній роботі було проведено аналіз та порівняння методів
машинного навчання для розв'язання задачі сегментації кратерів
на супутникових знімках.

У ході виконання даної роботи був проведений аналіз опублікованих
джерел за тематикою виявлення кратерів на супутникових
знімках, який показав, що тема є актуальною на сьогоднішній день.
У приведених роботах використовувалися різні підходи, а саме MPP, CNN, та
методи, засновані на використанні статистичних показників.
MPP показав свою ефективність у розв'язуванні цієї задачі, особливо
за умов, коли немає розміченої маски, або коли даних недостатньо для
тренування більш складної CNN моделі. Методи, засновані
на статистичних показниках теж потребують менше даних за CNN,
але потрібні знімки, розподілені у часі, що може ускладнити успішне використання
цього методу.
У той же час, якщо даних достатньо,
то CNN дає більш точні результати.

У ході формалізації завдання було дано визначення поняттю кратера,
описані вхідні дані, кроки методології та метрики, за якими будуть
порівнюватися моделі машинного навчання.

Наступним кроком дослідження був огляд моделей і функцій втрат.
Було досліджено три сучасні архітектури мереж: FPN, U-Net, та DeepLabv3 які вже
тривалий час показують свою ефективність у задачах сегментації зображень,
в основному, на медичних знімках. Досліджуваними функціями втрат були
Dice Loss і Focal Loss, які теж часто використовується у задачах сегментації. У ході
дослідження було виявлено, що дані інструменти гарно підійдуть для
сегментації кратерів на супутникових знімках.

Далі було підготовано датасет та імплементовано моделі. Датасет, який початково
складався із одного знімку та відповідної розміченої маски, було
поділено на випадкові патчі розміру 128x128, і обрані ті з них,
де маска складала хоча б 10\% від загальної картинки.
Таким чином, було отримано достатньо даних для навчання, валідації та
тестування обраних моделей. Імплементація моделей та функції втрат відбувалась за допомогою
бібліотек Segmentation Models PyTorch і PyTorch Lightning, які
значно спростили побудову та навчання моделей.

Наприкінці, було проведено порівняльний аналіз моделей машинного навчання.
Найбільш ефективною виявилась U-Net з використанням Dice Loss,
а найменш ефективною -- DeepLabv3 з використанням Focal Loss.
U-Net дає змогу точно визначити місцеположення кратерів, в той час
як інші дві моделі можуть надавати лише приблизні зони впливу,
а окремі кратери виділяються дуже рідко.
Також, варто зазначити, що моделі, які навчалися з Dice Loss
показують кращі і більш стабільні результати ніж ті,
що навчались з Focal Loss.

Практичні результати роботи були висвітлені на XXII Всеукраїнській
науково-практичній конференції студентів, аспірантів
та молодих вчених на базі Навчально-наукового Фізико-технічного
інституту Національного технічного університету України
<<Київський політехнічний інститут імені Ігоря Сікорського>>.