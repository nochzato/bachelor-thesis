%!TEX root = ../abstract.tex

\abstractUkr

Кваліфікаційна робота містить: 50 стор., 7 рисунків, 1 таблицю, 19 джерел.

Метою дослідження є уточнення методів
розв'язання задачі сегментації кратерів на супутникових знімках
на основі моделей машинного навчання.
Об'єктом дослідження є процес виявлення кратерів за супутниковими знімками.
Предметом дослідження є методи виявлення кратерів, сформованих
внаслідок вибухів бомб чи інших вибухових пристроїв, з використанням
сучасних технологій обробки зображень, супутникового зондування та
геоінформаційних систем.

Під час дослідження було проведено аналіз та порівняння
існуючих методів розв'язання задачі сегментації на супутникових
зображеннях. Використання результатів поставлених задач роботи
можуть бути застосовані для оцінки пошкоджень, визначення
найбільш вразливих зон для населення, тощо. Результати
дослідження показали, що модель U-Net з використанням Dice
Loss виявилася найбільш ефективною, здатною точно визначити
місцеположення кратерів. Також, виявлено, що моделі, навчені з
використанням Dice Loss, показали кращі та стабільніші результати
порівняно з моделями, навченими з використанням Focal Loss.

% наприкінці анотації потрібно зазначити ключові слова
Ключові слова: \MakeUppercase{машинне навчання, u-net, fpn, deeplabv3}

%%%% Рішенням кафедри з 2018 року ми прибираємо анотації російською мовою
% \abstractRus
%
%Русская аннотация должна быть точным переводом украинской (включая 
%статистику и ключевые слова).

\abstractEng

Qualification work contains: 50 pages, 7 figures, 1 table, 19 sources.

The purpose of the study is to refine methods for solving the problem of crater
segmentation on satellite images based on machine learning models. The object
of research is the process of detecting craters from satellite images. The
subject of the study is the methods of detecting craters formed as a result of
bombings or other explosive devices using modern image processing technologies,
satellite sensing and geographic information systems.

The study analyzed and compared existing methods for solving the segmentation
problem on satellite images. Using the results of the set tasks, the work can
be applied to damage assessment, identification of the most vulnerable areas
for the population, etc. The results of the study showed that the U-Net model
using Dice Loss proved to be the most effective, able to accurately determine
the location of craters. It was also found that the models trained using Dice
Loss showed better and more stable results compared to the models trained using
Focal Loss.

Keywords: \MakeUppercase{machine learning, u-net, fpn, deeplabv3}

% Не прибирайте даний рядок
\clearpage