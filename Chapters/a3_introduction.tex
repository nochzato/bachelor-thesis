%!TEX root = ../thesis.tex
% створюємо вступ
% \textbf{Актуальність дослідження.}
Повномасштабне вторгнення
в Україну призвело до значних ушкоджень сільськогосподарських угідь.
Деформація ґрунтового покриву внаслідок формування фортифікаційних
споруд (окопів), утворення кратерів від бомб, хімічне забруднення внаслідок
використання боєприпасів –- це лише короткий перелік наслідків вторгнення~\cite{golubtsov2023}.
Задача даної робити -- проаналізувати існуючі методи розв'язання задачі сегментації
на супутникових зображеннях і виявити найбільше ефективного.
Використання результатів поставлених задач роботи можуть бути застосовані для оцінки пошкоджень,
визначення найбільш вразливих зон для населення, тощо.

\textbf{Метою дослідження} є уточнення методів
розв'язання задачі сегментації кратерів на супутникових знімках
на основі моделей машинного навчання.
Для досягнення мети необхідно розв'язати наступні \textbf{задачі дослідження}:

\begin{enumerate}
    \item провести огляд опублікованих джерел за тематикою дослідження;
    \item формалізувати задачу сегментації кратерів на супутникових знімках
          та адаптувати до неї відомі методи розв'язання задачі сегментації;
    \item підготувати датасет для навчання моделей машинного навчання;
    \item провести порівняльний аналіз методів розв'язання задачі сегментації
          кратерів на супутникових знімках, ґрунтуючись на створеному датасеті.
\end{enumerate}

\emph{Об'єктом дослідження} є процес виявлення кратерів за супутниковими знімками.

\emph{Предметом дослідження} є методи виявлення кратерів, сформованих внаслідок вибухів
бомб чи інших вибухових пристроїв, з використанням сучасних технологій обробки зображень
та моделей машинного навчання.